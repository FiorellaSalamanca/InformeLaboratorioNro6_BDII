\section{Respuesta al ejecutar los siguientes comandos} 
\vspace{\baselineskip}
¿Qué sucede al ejecutar los siguientes comandos?
\begin{center}
	\includegraphics[width=8cm]{./Imagenes/1} 
\end{center}

\begin{itemize}
	\item STARTUP OPEN
\end{itemize}
\begin{adjustwidth}{0.40in}{0.0in}
	Una base de datos Oracle puede estar en uno de estos cuatro estados:
	OPEN: La base de datos está completamente funcional. Para ello se abren los archivos de datos y los Redo Log y se comprueba la consistencia de los datos.\\ \\
	Este es el valor por defecto para arrancar, montar y abrir una base datos.\\ \\
	Abrir la base de datos incluyendo las siguientes tareas:	
	\begin{itemize}
		\item[$*$] Apertura de los archivos de datos en línea.
		\item[$*$] Apertura de los archivos de registro de rehacer en línea.\\
\\
	\end{itemize}		
\end{adjustwidth}

\begin{itemize}
	\item STARTUP MOUNT 
\end{itemize}
\begin{adjustwidth}{0.40in}{0.0in}
	Una base de datos Oracle puede estar en uno de estos cuatro estados:
	MOUNT: Al estado anterior se añade la lectura de los archivos de control que permiten determinar cómo se ha de preparar la instancia. Se buscan los archivos de datos y los Redo Log, comprobando su existencia en las rutas marcadas por el archivo de control.\\ \\
	En este estado podemos conectar (como administradores) y realizar tareas como:
	\begin{itemize}
		\item[$*$] Cambio del nombre de los archivos de datos.
		\item[$*$] Activar el modo ARCHIVELOG.
		\item[$*$] Recuperación de la base de datos
		\item[$*$] En definitiva, tareas sobre los archivos de la base de datos ya que aun no se han abierto sus datos.	\\
\\
	\end{itemize}
	Arrancamos la base de datos montada, normalmente se usa en modo para tareas de mantenimiento.	
\end{adjustwidth}




	
	

\vspace{\baselineskip}
\vspace{\baselineskip}
